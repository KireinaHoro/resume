%-------------------------------------------------------------------------------
%	SECTION TITLE
%-------------------------------------------------------------------------------
\cvsection{Projects}

%-------------------------------------------------------------------------------
%	CONTENT
%-------------------------------------------------------------------------------
\begin{cventries}

%---------------------------------------------------------
  \cventry
    {Doctorate (with Prof.\ Dr.\ Timothy Roscoe)} % Job title
    {NetOS, Systems Group @ ETH Z\"urich} % Organization
    {Z\"urich, Switzerland} % Location
    {Since Dec.\ 2023} % Date(s)
    {
        \begin{cvitems}
            \item {Developing \emph{Lauberhorn}, a cache-coherent RPC NIC
                that is part of the OS}
            \item {Developing \emph{TxnLang}, a transaction-based
                intermediate language for HW formal verification}
            \item {Research focus: OS, networking, architecture, formal
                verification}
        \end{cvitems}
    }

%---------------------------------------------------------
  \cventry
    {Master Thesis (with Prof.\ Dr.\ Torsten Hoefler)} % Job title
    {Scalable Parallel Computing Lab (SPCL) @ ETH Z\"urich} % Organization
    {Z\"urich, Switzerland} % Location
    {Mar.\ 2023 - Sept.\ 2023} % Date(s)
    {
        \begin{cvitems}
            \item {Developed \emph{FPsPIN}, an FPGA prototype of the
                \emph{sPIN} in-network-compute paradigm}
            \item {Skills involved: Verilog, FPGA, systems programming in C}
        \end{cvitems}
    }

%---------------------------------------------------------
  \cventry
    {Semester Project (with Prof.\ Dr.\ Timothy Roscoe)} % Job title
    {NetOS, Systems Group @ ETH Z\"urich} % Organization
    {Z\"urich, Switzerland} % Location
    {Oct.\ 2022 - Feb.\ 2023} % Date(s)
    {
        \begin{cvitems}
            \item {Developed \emph{EFRI}, an OS-firmware interface for the
                \emph{Enzian} research computer}
            \item {Skills involved: systems programming in C, interface design}
        \end{cvitems}
    }

%---------------------------------------------------------
  \cventry
    {Undergraduate Research (with Prof.\ Yun Liang)} % Job title
    {Center for Energy-efficient Computing and Applications (CECA) @ PKU} % Organization
    {Beijing, China} % Location
    {Dec.\ 2017 - Jul.\ 2021} % Date(s)
    {
        \begin{cvitems}
            \item {Developed a prototype RISC-V-based accelerator platform on
                FPGAs}
            \item {Explored automatic compute intrinsic synthesis through MLIR
                and accelerator templates}
            \item {Skills involved: Chisel, systems programming in C, compiler
                design, C++, FPGA}
        \end{cvitems}
    }

%---------------------------------------------------------
  \cventry
    {Research Intern (with Prof.\ Babak Falsafi)} % Job title
    {Parallel Systems Architecture Lab (PARSA) @ EPFL} % Organization
    {Lausanne, Switzerland (remote)} % Location
    {Jul.\ 2020 - Jan.\ 2021} % Date(s)
    {
        \begin{cvitems}
            \item {Worked on a seL4 port for \emph{Midgard}, a new virtual
                memory scheme for terabyte-scale memory servers}
            \item {Skills involved: seL4, systems programming in C}
        \end{cvitems}
    }

%---------------------------------------------------------
  \cventry
    {Academic Collaboration (with Prof.\ Chenren Xu \& Dr.\ Pengyu Zhang)} % Job title
    {XG Lab @ Alibaba DAMO Academy} % Organization
    {Beijing, China} % Location
    {Sept.\ 2020 - Jan.\ 2021} % Date(s)
    {
        \begin{cvitems}
            \item {Developed the FPGA data capture and signal processing
                pipeline for a custom RFID localization system}
            \item {Skills involved: Verilog, FPGA, systems programming in C}
        \end{cvitems}
    }

%---------------------------------------------------------
  \cventry
    {Student Contributor (with Benda Xu <heroxbd>)} % Job title
    {Google Summer of Code @ Gentoo Foundation} % Organization
    {Remote} % Location
    {Apr.\ 2018 - Aug.\ 2018} % Date(s)
    {
        \begin{cvitems}
            \item {Developed \emph{SharkBait}, sandboxing Android with LXC
                inside native Gentoo on consumer hardware}
            \item {Skills involved: C, Android, Bash, build systems}
        \end{cvitems}
    }

%---------------------------------------------------------
  \cventry
    {Team Leader} % Job title
    {PKU Student Supercomputing Competition Team (PKUSC)} % Organization
    {Beijing, China} % Location
    {Nov.\ 2017 - Nov.\ 2020} % Date(s)
    {
        \begin{cvitems}
            \item {Built small clusters under tight power budget to solve
                super-computing challenges}
            \item {Skills involved: SysAdmin, C, C++, CUDA, Fortran}
        \end{cvitems}
    }

%---------------------------------------------------------
\end{cventries}

\cvsection{Work}

\begin{cventries}
    \cventry
    {Research Intern}
    {SenseTime}
    {Beijing, China}
    {Jun.\ 2019 - Dec.\ 2019}
    {
        \begin{cvitems}
            \item {Built the prototype of an in-house tensor compiler for
                deep-learning applications}
            \item {Skills involved: compiler design, C++}
        \end{cvitems}
    }
\end{cventries}

\cvsection{Teaching}

\begin{cventries}
    \cventry
    {Assistentz (head TA), Hilfsassistenz (HA)}
    {Advanced Operating Systems, ETH Z\"urich}
    {Z\"urich, Switzerland}
    {2022 - 2025}
    {}

    \cventry
    {Assistenz (head TA)}
    {System Programming and Computer Architecture, ETH Z\"urich}
    {Z\"urich, Switzerland}
    {2024}
    {}

    \cventry
    {Hilfsassistenz (HA)}
    {Computer Systems, ETH Z\"urich}
    {Z\"urich, Switzerland}
    {2022}
    {}

    \cventry
    {Teaching Assistant (TA)}
    {Computer Networks (Honor Track), Peking University}
    {Beijing, China}
    {Sept.\ 2020 - Feb.\ 2021}
    {
        \begin{cvitems}
            \item {Developed a lab assignment for students to implement their
                own NIC on FPGAs}
        \end{cvitems}
    }
\end{cventries}
